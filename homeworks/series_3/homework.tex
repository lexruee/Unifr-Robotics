\documentclass[12pt]{article}
\usepackage{url,graphicx,tabularx,array,geometry}
\usepackage[utf8]{inputenc}
\usepackage{amsmath}
\setlength{\parskip}{1ex} %--skip lines between paragraphs
\setlength{\parindent}{0pt} %--don't indent paragraphs
\usepackage{listing}
s
%-- Commands for header
\renewcommand{\title}[1]{\textbf{#1}\\}
\renewcommand{\line}{\begin{tabularx}{\textwidth}{X>{\raggedleft}X}\hline\\\end{tabularx}\\[-0.5cm]}
\newcommand{\leftright}[2]{\begin{tabularx}{\textwidth}{X>{\raggedleft}X}#1%
& #2\\\end{tabularx}\\[-0.5cm]}

%\linespread{2} %-- Uncomment for Double Space
\begin{document}

\title{Robotics Project  Autumn 2013}
\line
\leftright{\today}{Alexander Rüedlinger, 08-129-710, Group 01} %-- left and right positions in the header
\section*{Series 3}
\paragraph{a)} Done ;-)..

\paragraph{b)}
The constant value QUATER\_ROTATION = 9000 is used in a waiting / busy loop to rotate epuck 2 clockwise on the spot with NORM\_SPEED = 400. So the value QUATER\_ROTATION determines how long it takes for epuck 2 to make a quater rotation.

I guess the important thing to highlight here is the ratio:
\begin{equation}
\frac{9000}{400} = 22.5
\end{equation}
We can achieve the same rotation using only the left speed variable, but in this case we must set QUATER\_ROTATION = 18000.

\paragraph{c)} In the exercise file the distance is hardcoded. It's defined in the constant FORWARD which is set to 31 000. 

Instead to hardcode the distance we could use distance and color sensors. The epuck could detect the green pillar in the neighbourhood using color sensors and by a 360 degree rotation. If the green pillar is found the epuck stops and drives straightforward. While the epuck moves forward it computes the remaining distance. If a critical distance value is exceeded it stops.

\paragraph{d)} Epuck 2 informs epuck 1 by sending an external event called push\_info via emit push\_info. The epuck 1 listens to the event push\_info via onevent push\_info.

\paragraph{e)} Nothing to do. Works out of the box ;-).

\paragraph{f)} See aseba file.

\paragraph{g)} In order to find the green pillar in an unknown environment, we need a very intelligent exploration algorithm. This is task is similar to an ant that is looking for food.

A simple approach might be to transform this problem into a graph problem.
We could split the environment into squared fields. Each field consists of four vertices which are adjacent to vertices of neighboured squared fields.

In this case the problem consists of visiting each vertex and testing if a vertex is a green pillar. As an algorithm we could apply a breadth first search.

\end{document}
