\documentclass[12pt]{article}
\usepackage{url,graphicx,tabularx,array,geometry}
\usepackage[utf8]{inputenc}
\usepackage{amsmath}
\setlength{\parskip}{1ex} %--skip lines between paragraphs
\setlength{\parindent}{0pt} %--don't indent paragraphs

%-- Commands for header
\renewcommand{\title}[1]{\textbf{#1}\\}
\renewcommand{\line}{\begin{tabularx}{\textwidth}{X>{\raggedleft}X}\hline\\\end{tabularx}\\[-0.5cm]}
\newcommand{\leftright}[2]{\begin{tabularx}{\textwidth}{X>{\raggedleft}X}#1%
& #2\\\end{tabularx}\\[-0.5cm]}

%\linespread{2} %-- Uncomment for Double Space
\begin{document}

\title{Robotics Project  Autumn 2013}
\line
\leftright{\today}{Alexander Rüedlinger, 08-129-710, Group 01} %-- left and right positions in the header
\section*{Series 3}
\paragraph{a)} Done ;-)..

\paragraph{b)}
The constant value QUATER\_ROTATION = 9000 is used in a wait loop to rotate epuck 2 clockwise on the spot with NORM\_SPEED = 400.

I guess the important thing to highlight here is the ratio:
\begin{equation}
\frac{9000}{400} = 22.5
\end{equation}
We can achieve the same rotation using only the left speed variable, but in this case we must set QUATER\_ROTATION = 18000.

\paragraph{c)} In the exercise working file the distance is hardcoded. The distance is defined as the number of microseconds. A epuck has a built-in  dsPIC cpu (MIPS) running at  60Mhz. 60 millions cycles per second and 15 millions instructions per seconds.

When we compute the ratio between cpu clock speed and mips we get:
\begin{equation}
\frac{60Mhz}{15Mips} = 4 \text{ instructions per seconds} = 4000 \text{ instructions per } \mu s
\end{equation}
This computation seems to be correctly because all events are delayed by 4000 microseconds.

\paragraph{d)} Epuck 1 informs epuck 2 by sending an external even push\_info.

\paragraph{d)} Nothing to do. Works out of the box ;-).
\end{document}
